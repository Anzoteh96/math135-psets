\documentclass[11pt]{article}
\usepackage{amsmath,amssymb,pgf,tikz,fullpage,parskip,esvect}
\title{Challenges}
\author{ Anzo Teh Zhao Yang}
\date{December 21, 2016}
\usetikzlibrary{arrows}
\pagestyle{empty}

\begin{document}
\maketitle

\section {Extra practice 1.}
\begin{enumerate}
\item Let $n=3k$. If $k$ is odd then $k=2x+1$ for some $x\in\mathbb{Z}$. Now $2|n=3(2x+1)=6x+3=2(3x+1)+1$, so $2|1$, contradiction. Hence $k$ is even and write $k=2y$. Now $n=3k=3(2y)=6y=6\times y$ so $6|6y=n$.

\item $a^2+b^2+c^2-ab-bc-ca=\frac12((a-b)^2+(b-c)^2+(c-a)^2)\ge 0$, so $ab+bc+ca\le a^2+b^2+c^2=1$. On the other hand $0\le (a+b+c)^2=a^2+b^2+c^2+2ab+2bc+2ca=1+2ab+2bc+2ca$, so $2(ab+bc+ca)\ge -1$, or $ab+bc+ca\ge\frac{-1}{2}$.
\end{enumerate}

\section {Extra practice 2.}
\begin{enumerate}
\item We show that $p=3$. Indeed, if $3|p$ then we must have $p=3$, and if $3\nmid p$ we have $p\equiv \pm 1\pmod{3}$ so $p^2+2\equiv (\pm 1)^2+2=1+2=3\equiv 0\pmod {3}$. This means $3=p^2+2$, or $p=\pm 1$, contradiction. Hence $p=3$, and $p^3+2=3^3+2=29$ is a prime.

\item In first order number theory we could write $\exists L(\forall\epsilon [0<\epsilon \to(\exists\delta [0<\delta\land (\forall x [0<x+\delta \land x<\delta\to 3x+\epsilon<L\land 3x<L+\epsilon ])])])$. Notice that this is unnecessarily complicated and hardly readable but in first order number theory only constants, variables, parenthesis, $\exists, \forall, \land, \lor, <, =$ are allowed. We could have written as $\exists L(\forall \epsilon>0(\exists\delta>0 (|x|<\delta \to |3x-L|<\epsilon)))$.

We show that $L=0$ works (in fact, $L=0$ is the only number you should think of). For each $\epsilon$, choose $\delta=\frac{\epsilon}{3}$. Then $|x|<\delta\to |x|<\frac{\epsilon}{3}\to 3|x|<\epsilon to |3x|<\epsilon$.

\end{enumerate}

\section {Extra practice 3.}
\begin{enumerate}
\item Suppose such $a,b$ exist. From $a,b>0$ we have $a^4=b^4+b+1>b^4+b>b^4$, so $a>b$, and since $a,b\in\mathbb{Z}$ we have $a\ge b+1$ by the discreteness property of the integers. Now $b^4+b+1=a^4\ge (b+1)^4=b^4+4b^3+6b^2+4b+1$, or $4b^3+6b^2+3b\le 0$, contradicting that $b\ge 1$.

\item Let $a,b,c$ be the side lengths, with $c$ being the length of the hypothenuse. Given that $a^2+b^2=c^2$, we need to prove that one of $a,b,c$ is divisible by 3. Suppose not, that $3\nmid a, b, c$. Then $a\equiv \pm 1\pmod {3}$ and $a^2\equiv 1\pmod {3}$. Similarly $b^2\equiv c^2\equiv 1\pmod {3}$. Now from $2=1+1\equiv a^2+b^2=c^2\equiv 1\pmod {3},$ contradiction.
\end{enumerate}

\section {Extra practice 4.}
\begin{enumerate}
\item \emph {Existence.}
We go by strong induction on each positive integer $n$. For $n=1,2,3,4,5$ we can write them as $1,2,3,1+3,5$, respectively.

Now let this statement to be true for $1,2,\cdots , k-1$ for some $k\ge 6$. Since the Fibanacci sequence $F_i$ is unbounded and increasing, we can choose positive integer $p$ such that $p$ is the biggest positive integer with $F_p\le k.$ If $F_p=k$ we are done. Otherwise we have $F_p<k<F_{p+1}=F_p+F_{p-1}$, or $0<k-F_{p}<F_{p-1}$. Now by our induction hypothesis, $k-F{p}$ can be written as sum of distinct nonconsecutive Fibonacci numbers, namely $F_{i_1}+F_{i_2}+\cdots F_{i_x}$ for some $x\ge 1$, $i_1<i_2<\cdots i_x$ and $i_{j+1}-i_j\ge 2, \forall j\in [1,x-1]$. But from $k-F_p<F_{p-1}$ we have $i_x<p-1$. Therefore $k=F_{i_1}+F_{i_2}+\cdots F_{i_x}+F_p$ with $p-i_x\ge 2$.

\emph {Uniqueness.} We start with this claim:\\
Let $1<i_1<\cdots i_x$ be integers satisfying $i_{j+1}-i_j\ge 2$ for all $j\in [1,x-1].$ Then $F_{i_1}+F_{i_2}+\cdots F_{i_x}<F_{i_{x}+1}$.\\
Proof: we proceed by induction. If $x=1$ then we obviously have $F_{i_1}<F_{i_1+1}$ as $i_1\ge 2$ (recall that $F_1=F_2=1$ and $F_0=0$). Let us suppose that $F_{i_1}+F_{i_2}+\cdots F_{i_{x-1}}<F_{i_{x-1}+1}$. Then $F_{i_1}+F_{i_2}+\cdots F_{i_x}<F_{i_{x-1}+1}+F_{i_x}\le F_{i_{x}-1}+F_{i_x}=F_{i_{x}+1}$ since $i_{x-1}\le i_x-2$. This completes the induction proof.

Now we proceed with our main problem. Again we induct on $n$. For $n=1$ our only choice is $F_2=1$. Now let $1,2,\cdots k-1$ to be written uniquely as sum of distinct non-consecutive Fibonacci numbers for some $k\ge 2$. Let $p$ be the greatest positive integer with $F_p\le k$, so $F_p\le k<F_{p+1}$. Now let $F_{i_1}+F_{i_2}+\cdots F_{i_x}$ for some $x\ge 1$ and $F_{i_j}$ be distinct non-consecutive Fibonacci numbers, with $1<i_1<\cdots i_x$. If $i_x>p$ then $F_{i_x}\ge F_{p+1}>k$ which is impossible. If $i_x<p$ then from above $F_{i_1}+F_{i_2}+\cdots F_{i_x}<F_{i_{x}+1}\le F_p\le k,$ again a contradition. Hence $p=i_x$ and $F_{i_1}+F_{i_2}+\cdots F_{i_{x-1}}=k-F_p.$ If $k=F_p$ then we are done, since there is no way to write 0 as sum of positive integers. If $k>F_p$ then by induction hypothesis $k-F_p$ can be writen uniquely as sum of distinct non-consecutive Fibonacci numbers, so $F_{i_1},F_{i_2},\cdots F_{i_{x-1}}$ can be determined uniquely.

\end{enumerate}

\section{Extra practice 5.}
\begin{enumerate}
\item Let $x$ be any divisor of $a-1$. We claim that $x|n\Leftrightarrow x|\frac{a^n-1}{a-1}$. Indeed, since $a\equiv 1\pmod {x}$, we have $\frac{a^n-1}{a-1}=a^{n-1}+a^{n-2}+\cdots a+1\equiv 1+1+\cdots 1 (\text {$n$ times})=n\pmod {x}$. So $\frac{a^n-1}{a-1}\equiv 0$ iff $n\equiv 0$, in modulo $x$, justifying the claim.

Now if $x=\gcd (n, a-1)$ then $x|n, x|a-1$ and by the claim above, $x|\frac{a^n-1}{a-1}$ so $x$ is a commn divisor of $\frac{a^n-1}{a-1}$ and $a-1$, so $\gcd (n, a-1)\le\gcd (\frac{a^n-1}{a-1}, a-1).$ Similarly, if $x=\gcd (\frac{a^n-1}{a-1}, a-1)$ then $x|\frac{a^n-1}{a-1}$ and $x|a-1$, so by the claim above $x|n$. Therefore $x=\gcd (\frac{a^n-1}{a-1}, a-1)$ is the common divisor of $n$ and $a-1$, so $\gcd (\frac{a^n-1}{a-1}, a-1)\le \gcd(a-1, n)$. Combining the inequalties above yield $\gcd (\frac{a^n-1}{a-1}, a-1)= \gcd(a-1, n)$.

\item We claim that $\gcd (n,n+k)=k, \forall k\in [1,20]$ by inducting on $k$. Now $\gcd (n,n+k)=\gcd(n, (n+k)-n)=\gcd(n,k)\le k$. Therefore for base case $k=1$ we have $\gcd (n,n+1)\le 1$ and since 1 divides both $n+1$ and $n$ we have $\gcd (n, n+1)=1$. Now suppose that $\gcd (n, n+i)=i$ for some $1\le i\le 19$. Then $\gcd (n,n+i+1)>gcd(n,n+i)=i$ so $\gcd (n,n+1+1)\ge i+1$. On the other hand we have justifies that $\gcd (n,n+i+1)\le i+1$ as of above. Therefore $\gcd (n, n+i+1)=i+1$, completing the induction claim.

Now for all integers $k$ with $1\le k\le 20$ we have $\gcd (n,n+k)=k$ so $k|n, k|n+k$. This means $3|n, 7|n$ and since $\gcd (3,7)=1$, lcm(3,7)=$3\times 7=21$ so $21|n$ and $\gcd (n, n+21)=21>20=\gcd (n, n+20)$. Notice that the problem is true if we replace 21 with any number that is not a prime power.

\item First, we show that $2^x-1|2^{xy}-1$, $\forall x, y\ge 0$. Indeed, $2^x\equiv 1\pmod {2^x-1}$ so $2^{xy}=(2^x)^{y}\equiv 1^y\equiv 1\pmod {2^x-1}$. Therefore, since $\gcd (a,b)$ divides both $a$ and $b$, we have $2^{\gcd(a,b)}-1$ divides both $2^a-1$ and $2^b-1$, and therefore $2^{\gcd (a,b)}-1\le \gcd (2^a-1, 2^b-1)$.

Now first suppose that $a,b>0$. To prove the other direction we need a corollary: for all odd positive integers $x$, if $x|2^a-1$ and $x|2^b-1$ then $x|2^{\gcd (a,b)}-1$. Let $d$ be the minimum positive integer such that $x|2^d-1$ (this $d$ exists because $x|2^{\phi (x)}-1$ by Euler-Fermat theorem). We show that for all $k$, $x|2^k-1\Leftrightarrow d|k$. By Euclidean's remainder theorem we can write $k=bd+r$ with $0\le r<d$. Therefore $2^k=2^{bd+r} = 2^{bd}\cdot 2^r = (2^d)^b\cdot 2^r = \equiv 1^b\cdot 2^r=2^r\\pmod {x}$. If $r>0$, then by the minimality of $d$ we have $2^r\not\equiv 1\pmod {x}$ but if $r=0$, $2^r=1$. Thus $x|2^d-1\Leftrightarrow x|2^r-1\Leftrightarrow r=0\Leftrightarrow d|k$.

Now let's proceed with our claim, and here we let $x=\gcd (2^a-1, 2^b-1)$ (since 2 does not divide either of $2^a-1, 2^b-1$ for $a,b>0$, $\gcd (2^a-1, 2^b-1)$ is also odd, so the claim above applies to this $x$.) If we define $d$ as of above, the smallest positive integer with $x|2^d-1$, then from $x|2^a-1$ $x|2^b-1$ we have $d|a$ and $d|b$. This would imply $d|pa+qb$ for all $p,q\in\mathbb{Z}$, and since there exists such $p$ and $q$ with $pa+qb=\gcd (a,b)$ by Euclidean algorithm, $d|\gcd (a,b)$. But this implies $x=\gcd (2^a-1, 2^b-1)|2^{\gcd (a,b)}-1$, so $\gcd (2^a-1, 2^b-1)\le 2^{\gcd (a,b)}-1$. Summing the two inequalities we have $2^{\gcd (a,b)}-1= \gcd (2^a-1, 2^b-1)$ for $a,b$ positive.

In the case $a=0$ then $2^{\gcd (a,b)}-1=2^{\gcd (0,b)}-1=2^b-1=\gcd (0, 2^b-1)=\gcd (2^0-1, 2^b-1)=\gcd (2^a-1, 2^b-1)$. The case $b=0$ is completely analogous.

\end{enumerate}

\section {Extra practice 6.}
\begin{enumerate}
\item (a) Yes, since 1+2+3+6=12=$2\times 6$.\\
(b) No, since 1+7=8$\neq 14$.\\
(c) Since $2^k-1$ is prime, all divisors of $n=2^{k-1}(2^k-1)$ can be written in the form of $ab$ with $a=2^i$ for some $i$ with $0\le i\le k-1$ and $b\in\{1, 2^k-1\}$, due to the theorem of prime factorization. Therefore, the sum of divisors is\\ $1+(2^k-1)+2+2(2^k-1)+\cdots + 2^{k-1}+(2^{k-1})(2^k-1)$\\
$=(1+2^k-1)+2(1+2^k-1)+\cdots 2^{k-1}(1+2^k-1)$\\
$=(1+2+\cdots 2^{k-1})(1+2^k-1)$\\
$=(2^k-1)(2^k)$\\
$=2(2^{k-1})(2^k-1)$.

Hence this number is perfect.

\item We denote $p_1, p_2,\cdots p_k$ as all the primes dividing either $a$ or $b$ or both. By theorem of prime factorization, we can write $a=\displaystyle\prod_{i=1}^{k} p_i^{a_i}$ and $b=\displaystyle\prod_{i=1}^{k} p_i^{b_i}$. Therefore $\gcd (a^n, b^n)$ = $\gcd ((\displaystyle\prod_{i=1}^{k} p_i^{a_i})^n, (\displaystyle\prod_{i=1}^{k} p_i^{b_i})^n)$ = $\gcd (\displaystyle\prod_{i=1}^{k} p_i^{na_i}, \displaystyle\prod_{i=1}^{k} p_i^{nb_i}) = \displaystyle\prod_{i=1}^{k} p_i^{\min (na_i, nb_i)}$ = $\displaystyle\prod_{i=1}^{k} p_i^{n\min (a_i, b_i)}$ = $(\displaystyle\prod_{i=1}^{k} p_i^{\min (a_i, b_i)})^n$ = $(\gcd (a,b))^n.$ Notice that we used the fact that $\min (nx, ny)=n\min (x,y)$ for $n\ge 0$ since if $x\le y$ then $nx=ny=n(x-y)\le 0$ so $nx\le ny$ and $\min (nx, ny)=nx=n\min (x,y)$. Similarly if $x\ge y$ then $\min (nx, ny)=ny=n\min (x,y)$.
\end{enumerate}

\section {Extra practice 7.}
\begin{enumerate}
\item Fo clarity we denote $b_i$ as the digit appended on the end of $a_{i-1}$ to orm $a_i$. We split into several senarios:\\
\emph{Scenario 1.} If $b_i\in\{0,2,4,6,8\}$ for infinitely many $i$, then $a_i$ is even for such $i$, hence composite. If $b_i$ is 0 or 5 for infinitely many $i$ then for such $i$, $5|a_i$, hence composite.

\emph {Scenario 2.} Suppose that scenario 1 didn't happen. Then there exists an $N$ such that for all $k\ge N$ we have $b_k\in\{1,3,7\}$. Now further assume that for this scenario, $b_i\in\{1,7\}$ infinitely many times. Then there exists sequence $N\le c_1<c_2<\cdots$ such that for all $i\in\mathbb{N}$, we have $b_{c_i}\in\{1,7\}$, so $b_{c_i}\equiv 1\pmod {3}$. Now if $c_x<j<c_{x+1}$ for some $j$, $b_j=3$ by our definition of this sequence. Also let $a_{c_1}\equiv g\pmod {3}$ for some $g\in\{0,1,2\}$. Now $a_{c_2}\equiv a_{c_1}+b_{c_1+1}+b_{c_1+2}+\cdots + b_{c_2}\equiv a_{c_1}+3+3+\cdots +3+1\equiv a_{c_1}+1$ (we used the fact that for every integer $n$, $n$ is equal to its sum of digits in modulo 3). Inductively, $a_{c_{i+1}}\equiv a_{c_i}+1\pmod{3}$ so $a_{c_{i+1}}\equiv a_{c_1}+i\equiv g+i\pmod {3}.$ Now for all $i=3k-g$ for $g\ge 1$, $a_{c_i}$ is divisible by 3, so is composite.

\emph {Scenario 3.} Suppose that both scenarios 1 and 2 didn't happen, then there exists $N$ such that for all $k\ge N$, $b_k=3$. Let $m=a_N$, which ends with digit 3. If $3|m$ then $3|a_k$ for all $k\ge N$, so let's assume $3\nmid m$. We can see that $\gcd (10, m)=1$ since $m$ is divisible by neither 2 nor 5. Now, by Euler's theorem, for all positive integers $j$, $10^{j\phi(m)}=(10^{\phi (m)})^j\equiv 1^j\equiv 1\pmod {m}$, and we know the number $\underbrace{33\cdots 3}_{j\phi(m)\text{ times}}=3(\frac{10^{j\phi(m)}-1}{9})=\frac{10^{j\phi(m)}-1}{3}.$  is divisible by $m$ since $m|10^{j\phi(m)}-1$ and $\gcd (m,3)=1$. Now for all $j$, $a_{N+j\phi(m)}=a_N(10^{j\phi(m)})+\frac{10^{j\phi(m)}-1}{3}$ = $m(10^{j\phi(m)})+\frac{10^{j\phi(m)}-1}{3}$ is divisible by $m$, hence is composite. Q.E.D.

\item We need this identity: The highest power of 2 that divides $3^{2^k}-1$ is $k+2$ for $k\ge 1$. Let's proceed by induction. If $k=1$ then $3^{2^1}-1=8=2^3=2^{1+2}$. Now suppose that the highest power of 2 dividing $3^{2^p}-1$ is $p+2$ for some $p\ge 1.$ Then by induction hypothesis $3^{2^p}-1=c\cdot 2^{p+2}$ for some odd positive integer $c$. Now $3^{2^{p+1}}-1=(3^{2^p}-1)(3^{2^p}+1)=c\cdot 2^{p+2}\cdot(c\cdot 2^{p+2}+2) = c^2\cdot 2^{2p+4}+ c\cdot 2^{p+3}$. Now $\frac{c^2\cdot 2^{2p+4}+ c\cdot 2^{p+3}}{2^{p+3}}$ = $c^2\cdot 2^{p+1}+c.$ Since $p\ge 1$, $c^2\cdot 2^{p+1}$ is even but $c$ is odd, so $c^2\cdot 2^{p+1}+c$ = $\frac{c^2\cdot 2^{2p+4}+ c\cdot 2^{p+3}}{2^{p+3}}$ is an odd integer, and thus the highest power of 2 dividing $3^{2^{p+1}}-1$, completing the claim.

Now for the main problem we proceed by inducting on $k$. For $k=1,2,3$ we can choose $n=1$, so that $3+5=8$ is divisible by 2,4, and 8. Now suppose that for some $k\ge 3$, we can find $n_k$ such that $2^k|3^{n_k}+5$. We want to prove that we can find $n_{k+1}$ such that $2^{k+1}|3^{n_{k+1}}+5.$ If $2^{k+1}|3^{n_k}+5$ then we can choose $n_{k+1}=n_k$. Otherwise, we can write $3^{n_k}+5=c\cdot 2^k$ for some odd $c$. Now choose $n_{k+1}=n_k+2^{k-2}$. Recall that by above the highest power of 2 dividing $3^{2^{k-2}}-1$ is $k$, so we can write $3^{2^{k-2}}-1$ as $d\cdot 2^k$ for some odd $d$. Therefore, $3^{n_{k+1}}+5=3^{n_k+2^{k-2}}+(3^{n_k})(3^{2^{k-2}})+5=(c\cdot 2^k-5)(d\cdot 2^k+1)+5=cd\cdot 2^{2k}-5d\cdot 2^k+c\cdot 2^k -5 +5 = cd\cdot 2^{2k}+(c-5d)\cdot 2^k = (2^{k+1})(cd\cdot 2^{k-1}+\frac{c-5d}{2})$. Now since $k\ge 3$, $2^{k-1}$ is an integer and since $c$ and $5d$ are both odd, $c-5d$ is even and therefore $\frac{c-5d}{2}$ is an integer. Therefore $cd\cdot 2^{k-1}+\frac{c-5d}{2}$ is an integer and $2^{k+1}|3^{n_{k+1}}+5$, completing the induction proof.
\end{enumerate}

\section {Extra practice 9.}
\begin{enumerate}
\item Write $z=a+bi$ and $w=c+di$ for $a,b,c,d$ real. Then:

(a) $|z+w|=|(a+c)+(b+d)i|=\sqrt{(a+c)^2+(b+d)^2}$ while $|z|+|w|=\sqrt{a^2+b^2}+\sqrt{c^2+d^2}$. By Cauchy-Schrawz inequality we have $(a^2+b^2)(c^2+d^2)\ge (ac+bd)^2$. Therefore\\ 
$(\sqrt{(a+c)^2+(b+d)^2})^2$\\
$=(a+c)^2+(b+d)^2$\\
$=a^2+b^2+c^2+d^2+2ac+2bd$\\
$\le a^2+b^2+c^2+d^2+2|ac+bd|$\\
$= a^2+b^2+c^2+d^2+2\sqrt{(ac+bd)^2}$\\
$\le a^2+b^2+c^2+d^2+2\sqrt{(a^2+b^2)(c^2+d^2)}$\\
$=(\sqrt{a^2+b^2}+\sqrt{c^2+d^2})^2$,\\
so $|z+w|=|(a+c)+(b+d)i|=\sqrt{(a+c)^2+(b+d)^2}\le \sqrt{a^2+b^2}+\sqrt{c^2+d^2}=|z|+|w|$.

(b) The right inequality is almost similar as above. For the left inequality, by (a) we have $|z|=|w+(z-w)|\le |w|+|z-w|$ so $|z|-|w|\le |z-w|$. Also $|w|=|z+(w-z)|\le |z|+|w-z|=|z|+|z-w|$ (as $|a|=|-a|$ for all $a\in\mathbb{C})$ so $|w|-|z|\le |z-w|$. Summing up both inequalities yield $||z|-|w||\le |z-w|$.

\item On the Cartesian plane, denote $A, B, C$ as the coordinate corresponding to $a,b,c$ on complex plane. Then in vector form $b-a=\vv{AB}$, $a-c=\vv{CA}$ and $c-b=\vv{BC}$. It suffices to prove that $A, B, C$ either all coincide or are the vertices of an equilateral triangle. Observe $a-c$ and $c-b$ cannot be zero (otherwise the quotient may not be defined) so $\frac{b-a}{c-b}=\frac{a-c}{c-b}\neq 0$, and $b-a$ cannot be zero too. Thus no two point coincide.

Let's consider the case where $A,B,C$ are not collinear. Now we show that $\angle BAC=\angle ACB$. In subsequent solution we will talk about $\arg$ of vector in modulo $2\pi$. Now, $\arg (b-a)-\arg (a-c)$ = $\arg (\frac{b-a}{a-c})$ = $\arg (\frac{a-c}{c-b})$ = $\arg (a-c)-\arg (c-b)$. Also notice that $\arg (b-a)-\arg (a-c)$ is the counterclockwise angle needed to make vector $\vv{CA}$ parallel to (and heading the same direction with) $\vv{AB}$. Now, if $A,B,C$ are in counterclowkwise order then $\arg (b-a)-\arg (a-c)= \pi-\angle BAC$ and $\arg (a-c)-\arg (c-b)=\pi-\angle ACB$. Therefore $\angle BAC=\angle ACB$. If $A,B,C$ are in clockwise order then $\arg (b-a)-\arg (a-c)= \pi+\angle BAC$ and $\arg (a-c)-\arg (c-b)=\pi+\angle ACB$. Therefore $\angle BAC=\angle ACB$. Now we have $|BC|=|AB|$, and $\frac{|AB|}{|CA|}=\frac{|CA|}{|BC|}$, or $|CA|^2=|AB|\cdot |BC|=|AB|\cdot |AB|=|AB|^2$, so $|CA|=|AB|$. 

If $A,B,C$ are collinear (which holds vacuously when any two of them coincide) then $\arg (b-a)-\arg (a-c)$ and $\arg (a-c)-\arg (c-b)$ are both 0 or $\pi$. If they are 0 then $\vv{AB}, \vv{CA}, \vv{BC}$ are all pointing to the same direction, which is impossible. If they are $\pi$, then $\vv{AB}$ and $\vv{BC}$ are pointing at the same direction while $\vv{CA}$ pointing to the opposite direction. This means $B$ is in between $A$ and $C$ and we have $|CA|=|AB|+|BC|$. Now $1>\frac{|AB|}{|AB+BC|}=\frac{|AB|}{|CA|}=\frac{|CA|}{|BC|}=\frac{|AB+BC|}{|CA|}>1$ since we assumed that $|CA|, |AB|, |BC|>0$, contradiction.

\end{enumerate}

\end{document}